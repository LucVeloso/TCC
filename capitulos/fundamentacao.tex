\chapter{Fundamentação Teórica}
\label{cap:fundamentacao}

\section{Conceitos de Medição de Energia Elétrica}

A medição de energia elétrica em ambientes industriais é um pilar fundamental para a gestão energética e para o cumprimento de normas como a ISO 50001 \cite{iso50001}. O monitoramento contínuo e preciso de grandezas elétricas permite a identificação de ineficiências, a alocação correta de custos e a otimização do consumo \cite{mamede2017}.

\subsection{Grandezas Elétricas Monitoradas (Potência Ativa, Reativa, Fator de Potência)}

Em sistemas de corrente alternada (CA), a potência elétrica é classificada em três componentes principais, cuja compreensão é essencial para a análise do desempenho energético de uma planta industrial:

\begin{itemize}
    \item \textbf{Potência Ativa ($P$):} Medida em Watts (W) ou quilowatts (kW), é a energia efetivamente convertida em trabalho útil, como o movimento de motores ou o aquecimento \cite{mamede2017}. É a potência que realiza o trabalho e, portanto, a que é faturada pela concessionária de energia.
    \item \textbf{Potência Reativa ($Q$):} Medida em Volt-Ampère Reativo (VAr) ou quilovolt-ampère reativo (kVAr), é a potência necessária para criar e manter os campos eletromagnéticos em equipamentos indutivos, como motores, transformadores e reatores \cite{kosow1982}. Embora não realize trabalho útil, é indispensável para o funcionamento desses equipamentos e sua circulação na rede pode sobrecarregar o sistema.
    \item \textbf{Potência Aparente ($S$):} Medida em Volt-Ampère (VA) ou quilovolt-ampère (kVA), é a soma vetorial da Potência Ativa e da Potência Reativa. Representa a potência total que o sistema de distribuição deve fornecer para atender à demanda de carga \cite{mamede2017}.
\end{itemize}

O \textbf{Fator de Potência ($FP$)} é a razão entre a Potência Ativa e a Potência Aparente ($FP = P/S$). Ele é um indicador crucial da eficiência com que a energia elétrica está sendo utilizada \cite{medeiros2021}. Um baixo fator de potência indica um consumo excessivo de Potência Reativa, o que pode levar a multas por parte da concessionária e a perdas de energia na rede interna da indústria \cite{prado2021}. O monitoramento contínuo dessas grandezas é a base para a implementação de ações de correção e otimização energética.

\subsection{Energia Gerada e Regenerativa em Linhas de Teste}

O conceito de \textbf{energia regenerativa} é amplamente conhecido no setor automotivo, especialmente em veículos elétricos e híbridos, onde o motor elétrico atua como um gerador durante a frenagem ou desaceleração, convertendo a energia cinética do veículo de volta em energia elétrica \cite{azevedo2020}.

No contexto industrial, e especificamente nas linhas de montagem de veículos, a presença de \textbf{linhas de teste} para motores ou veículos completos pode gerar um fenômeno similar. Durante testes de desempenho ou simulações de frenagem, o motor do veículo, quando acoplado a um sistema de dinamômetros ou freios regenerativos, pode funcionar como um gerador, injetando energia de volta na rede elétrica da fábrica \cite{andrade2019}.

A medição e quantificação dessa energia gerada e regenerativa é um diferencial importante para o balanço energético do setor. A capacidade de monitorar o fluxo bidirecional de energia (consumo e geração) permite o cálculo mais preciso do consumo líquido de cada linha de montagem e a identificação de oportunidades para otimizar o aproveitamento dessa energia, alinhando-se aos princípios de sustentabilidade e eficiência energética da ISO 50001 \cite{iso50001}.


\section{Conceitos de Medição de Energia Elétrica}

A medição de energia elétrica em ambientes industriais é um pilar fundamental para a gestão energética e para o cumprimento de normas como a ISO 50001 (ABNT, 2018). O monitoramento contínuo e preciso de grandezas elétricas permite a identificação de ineficiências, a alocação correta de custos e a otimização do consumo (MAMEDE FILHO, 2017).

\subsection{Grandezas Elétricas Monitoradas (Potência Ativa, Reativa, Fator de Potência)}

Em sistemas de corrente alternada (CA), a potência elétrica é classificada em três componentes principais, cuja compreensão é essencial para a análise do desempenho energético de uma planta industrial:

\begin{itemize}
    \item \textbf{Potência Ativa ($P$):} Medida em Watts (W) ou quilowatts (kW), é a energia efetivamente convertida em trabalho útil, como o movimento de motores ou o aquecimento (MAMEDE FILHO, 2017). É a potência que realiza o trabalho e, portanto, a que é faturada pela concessionária de energia.
    
    \item \textbf{Potência Reativa ($Q$):} Medida em Volt-Ampère Reativo (VAr) ou quilovolt-ampère reativo (kVAr), é a potência necessária para criar e manter os campos eletromagnéticos em equipamentos indutivos, como motores, transformadores e reatores (KOSOW, 1982). Embora não realize trabalho útil, é indispensável para o funcionamento desses equipamentos e sua circulação na rede pode sobrecarregar o sistema.
    
    \item \textbf{Potência Aparente ($S$):} Medida em Volt-Ampère (VA) ou quilovolt-ampère (kVA), é a soma vetorial da Potência Ativa e da Potência Reativa. Representa a potência total que o sistema de distribuição deve fornecer para atender à demanda de carga (MAMEDE FILHO, 2017).
\end{itemize}

O \textbf{Fator de Potência ($FP$)} é a razão entre a Potência Ativa e a Potência Aparente ($FP = P/S$). Ele é um indicador crucial da eficiência com que a energia elétrica está sendo utilizada (MEDEIROS, 2021). Um baixo fator de potência indica um consumo excessivo de Potência Reativa, o que pode levar a multas por parte da concessionária e a perdas de energia na rede interna da indústria (PRADO, 2021). O monitoramento contínuo dessas grandezas é a base para a implementação de ações de correção e otimização energética.

\subsection{Energia Gerada e Regenerativa em Linhas de Teste}

O conceito de \textbf{energia regenerativa} é amplamente conhecido no setor automotivo, especialmente em veículos elétricos e híbridos, onde o motor elétrico atua como um gerador durante a frenagem ou desaceleração, convertendo a energia cinética do veículo de volta em energia elétrica (AZEVEDO; GUARNIERI; BOCCATO, 2020).

No contexto industrial, e especificamente nas linhas de montagem de veículos, a presença de \textbf{linhas de teste} para motores ou veículos completos pode gerar um fenômeno similar. Durante testes de desempenho ou simulações de frenagem, o motor do veículo, quando acoplado a um sistema de dinamômetros ou freios regenerativos, pode funcionar como um gerador, injetando energia de volta na rede elétrica da fábrica (ANDRADE, 2019).

A medição e quantificação dessa energia gerada e regenerativa é um diferencial importante para o balanço energético do setor. A capacidade de monitorar o fluxo bidirecional de energia (consumo e geração) permite o cálculo mais preciso do consumo líquido de cada linha de montagem e a identificação de oportunidades para otimizar o aproveitamento dessa energia, alinhando-se aos princípios de sustentabilidade e eficiência energética da ISO 50001 (ABNT, 2018).

\section{Instrumentação Industrial}

A instrumentação industrial desempenha um papel fundamental na modernização das plantas produtivas, permitindo a coleta precisa de dados para a gestão de ativos e eficiência energética. No contexto da Indústria 4.0, a integração de dispositivos de medição inteligentes com controladores lógicos programáveis possibilita não apenas o monitoramento em tempo real, mas também a análise preditiva e a otimização de processos \cite{silva2023}.

\subsection{Multimedidores Siemens Sentron PAC 3200}

O multimedidor SENTRON PAC3200 é um dispositivo multifuncional projetado para a medição e monitoramento de grandezas elétricas em sistemas de distribuição de energia de baixa tensão. Este equipamento é capaz de registrar mais de 50 parâmetros elétricos, incluindo tensões de fase e de linha, correntes, potências (ativa, reativa e aparente), fator de potência e frequência \cite{siemens2017pac}.

De acordo com o manual técnico do fabricante, o PAC3200 destaca-se pela sua alta precisão na medição de energia ativa (Classe 0.5S conforme IEC 62053-22) e pela versatilidade de comunicação, suportando protocolos como Modbus TCP de forma nativa e outros protocolos industriais via módulos de expansão \cite{siemens2017pac}. Em aplicações de gestão energética baseadas na ISO 50001, o uso desses dispositivos permite a rastreabilidade do consumo e a identificação de pontos de desperdício, sendo essencial para a composição de indicadores de desempenho energético (EnPIs) \cite{peixoto2023}.

\subsection{Controladores Lógicos Programáveis (CLP) Siemens S7-1200}

O SIMATIC S7-1200 é um controlador modular compacto, posicionado pela Siemens como uma solução versátil para automação básica e intermediária. Sua arquitetura integra uma CPU com interfaces de comunicação (PROFINET/Ethernet), entradas e saídas digitais e analógicas, além de suporte a funções tecnológicas como controle de movimento e contagem rápida \cite{siemens2024s71200}.

No âmbito do monitoramento energético, o S7-1200 atua como um concentrador de dados e elemento de controle. Através do protocolo Modbus TCP ou OPC UA, o CLP pode realizar a leitura determinística dos dados provenientes dos multimedidores PAC3200, processando essas informações localmente para tomadas de decisão rápidas ou encaminhandoas para sistemas supervisórios e bancos de dados \cite{lino2025}. A flexibilidade do S7-1200 permite a implementação de lógicas de controle de demanda, onde cargas não prioritárias podem ser desligadas automaticamente para evitar multas por excesso de demanda contratada, alinhando a operação da planta aos objetivos de eficiência energética \cite{peixoto2023}.


\section{Comunicação e Sistemas Supervisórios}

A comunicação eficiente entre dispositivos de campo e sistemas de gestão é o alicerce da digitalização industrial. No contexto da Indústria 4.0, a interoperabilidade deixa de ser apenas uma conveniência técnica para se tornar um requisito estratégico, permitindo que dados brutos de sensores e medidores sejam transformados em informações acionáveis para a tomada de decisão \cite{melo2020}.

\subsection{Protocolo OPC UA}

O \textit{Open Platform Communications Unified Architecture} (OPC UA) é um protocolo de comunicação industrial independente de plataforma e orientado a serviços, amplamente reconhecido como o padrão de interoperabilidade para a Indústria 4.0 \cite{sousa2024}. Diferente de protocolos tradicionais como o Modbus, que se limitam à transferência de registros de dados, o OPC UA oferece um modelo de informação rico, onde os dados são acompanhados de metadados que descrevem seu significado e contexto \cite{ladegourdie2022}.

Uma das principais vantagens do OPC UA é a sua robustez em termos de segurança, integrando mecanismos de autenticação, autorização e criptografia diretamente em sua pilha de comunicação \cite{melo2020}. Além disso, sua arquitetura cliente-servidor permite que múltiplos clientes (como sistemas supervisórios, bancos de dados ou aplicações em nuvem) acessem simultaneamente os dados de um servidor OPC UA integrado em dispositivos como o CLP Siemens S7-1200, garantindo uma integração vertical fluida entre o chão de fábrica e os sistemas corporativos \cite{martins2023}.

\subsection{Arquitetura SCADA e Sistemas de Aquisição de Dados (DAQs)}

Os sistemas de Controle Supervisório e Aquisição de Dados (\textit{Supervisory Control and Data Acquisition} - SCADA) são arquiteturas de software projetadas para monitorar e controlar processos industriais complexos em tempo real \cite{fortinet2024}. Um sistema SCADA típico é composto por quatro elementos principais: a interface homem-máquina (HMI), o sistema de supervisão, as unidades terminais remotas (RTUs) ou controladores lógicos programáveis (CLPs), e a infraestrutura de comunicação \cite{cast4it2024}.

No âmbito do monitoramento energético, os sistemas SCADA atuam como Sistemas de Aquisição de Dados (DAQs) centralizados, responsáveis por coletar variáveis elétricas de diversos pontos da planta, processar cálculos de indicadores de desempenho (EnPIs) e gerar alarmes em caso de anomalias de consumo \cite{pma2025}. A evolução dessas arquiteturas para o modelo de "SCADA baseado em Web" ou "Cloud SCADA" permite que gestores de energia acessem dashboards de consumo de qualquer local, facilitando o cumprimento dos requisitos de monitoramento contínuo exigidos pela norma ISO 50001 \cite{checklist2025}.

\section{Indicadores de Desempenho Energético (EnPIs)}

A medição e o monitoramento do desempenho energético são requisitos centrais para qualquer organização que busque a certificação na norma ISO 50001. Para que a gestão seja eficaz, é necessário traduzir dados complexos de consumo em métricas compreensíveis e comparáveis, conhecidas como Indicadores de Desempenho Energético (\textit{Energy Performance Indicators} - EnPIs) \cite{iso50001}.

\subsection{Definição e Importância conforme ISO 50001 e ISO 50006}

De acordo com a norma ABNT NBR ISO 50001:2018, um EnPI é definido como um "valor ou medida quantitativa do desempenho energético, conforme definido pela organização" \cite{iso50001}. Enquanto a ISO 50001 estabelece os requisitos gerais para o sistema de gestão, a norma complementar ISO 50006 fornece diretrizes específicas para o estabelecimento, utilização e manutenção desses indicadores, garantindo que eles sejam representativos da realidade operacional da planta \cite{andersson2021}.

A importância dos EnPIs reside na sua capacidade de demonstrar a melhoria do desempenho energético ao longo do tempo. Sem indicadores bem definidos, uma organização pode observar uma redução no consumo total de energia devido a uma queda na produção, e não necessariamente devido a um ganho de eficiência \cite{bruni2021}. Portanto, os EnPIs devem ser normalizados em relação a variáveis relevantes, como volume de produção, horas de operação ou condições climáticas, para permitir uma comparação justa entre diferentes períodos \cite{shim2018}.

\subsection{Tipos de Indicadores e Aplicação no Setor Automotivo}

Os indicadores de desempenho energético podem ser classificados em diferentes níveis de complexidade, dependendo do objetivo da análise \cite{iso50006}:

\begin{itemize}
    \item \textbf{Valor de Consumo Absoluto:} Medição direta da energia consumida (ex: kWh total por mês). É útil para contabilidade e faturamento, mas limitado para análise de eficiência.
    \item \textbf{Indicadores de Intensidade Energética:} Relação entre o consumo de energia e uma métrica de produção (ex: kWh por veículo produzido). É o indicador mais comum no setor automotivo para monitorar a eficiência das linhas de montagem \cite{nissan2014}.
    \item \textbf{Modelos Estatísticos (Regressão Linear):} Utilizam ferramentas estatísticas para prever o consumo esperado com base em múltiplas variáveis. A diferença entre o consumo real e o previsto indica o ganho ou perda de eficiência \cite{shim2018}.
\end{itemize}

No contexto de uma planta automotiva como a Stellantis, a aplicação de EnPIs por linha de montagem ou por setor (pintura, funilaria, montagem final) permite identificar os "Usos Significativos de Energia" (USEs). Isso possibilita que a equipe de manutenção e gestão foque seus esforços de otimização nos processos que possuem maior impacto no custo final e na pegada de carbono da organização \cite{peixoto2023}.


% Conteúdo original do arquivo fundamentacao.tex (mantido como referência para o usuário)
% Após introduzir o contexto da eficiência energética industrial e a necessidade de monitoramento conforme a ISO 50001, esta seção apresenta os conceitos fundamentais relacionados à gestão de energia, aquisição de dados industriais e diagnóstico de falhas em motores elétricos, os quais embasam o desenvolvimento do sistema proposto.

% \section{Gestão e Eficiência Energética}
% A gestão energética tem como objetivo otimizar o uso dos recursos elétricos de modo a reduzir perdas e custos operacionais. No contexto industrial, o consumo elétrico está fortemente associado ao uso de motores e equipamentos de grande porte, o que torna essencial o acompanhamento contínuo de indicadores de desempenho energético.

% A norma ISO 50001 estabelece diretrizes para implementação de sistemas de gestão de energia (SGE), baseados no ciclo PDCA (Planejar, Executar, Verificar e Agir). O cumprimento dessa norma exige medições precisas de consumo, definição de indicadores e documentação de melhorias contínuas no desempenho energético.

% \section{Monitoramento e Aquisição de Dados Industriais}
% O monitoramento contínuo é um dos pilares para atender aos requisitos da ISO 50001. Em ambientes industriais, ele é realizado por dispositivos de medição, como analisadores de energia e controladores programáveis, integrados via protocolos de comunicação.

% Sistemas SCADA (Supervisory Control and Data Acquisition) permitem a supervisão de processos e o registro de variáveis elétricas em tempo real. A integração desses sistemas com tecnologias IoT possibilita o armazenamento e análise remota dos dados energéticos.

% Os principais protocolos utilizados em redes industriais são Modbus, Profinet e OPC UA, que garantem interoperabilidade entre equipamentos de diferentes fabricantes e compatibilidade com plataformas de análise.

% \section{Motores Elétricos e Diagnóstico de Falhas}
% Os motores de indução trifásicos são responsáveis por uma parcela significativa do consumo energético industrial. O desempenho desses motores é diretamente afetado por condições elétricas e mecânicas, sendo o diagnóstico precoce de falhas essencial para evitar paradas não programadas.

% As falhas mais comuns incluem curto-circuitos entre espiras, desgaste de rolamentos, desalinhamentos e desequilíbrios de fase. Técnicas de diagnóstico como a Análise de Assinatura de Corrente (MCSA) e o uso da Transformada Rápida de Fourier (FFT) permitem identificar variações nos sinais de corrente e vibração associadas a essas falhas.

% \section{Instrumentação e Dispositivos de Medição}
% Para o monitoramento elétrico, utilizam-se sensores de corrente, tensão e temperatura, além de medidores inteligentes que consolidam as informações de consumo e qualidade de energia. Dispositivos como o Siemens Sentron PAC3200 possibilitam a leitura detalhada de parâmetros como potência ativa e reativa, fator de potência e distorção harmônica total (THD).

% Esses dados, quando tratados e analisados, permitem tanto o acompanhamento do desempenho energético quanto a identificação de condições anormais em equipamentos, alinhando o sistema às práticas de manutenção preditiva e gestão energética exigidas pela ISO 50001.
