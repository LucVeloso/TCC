\chapter{Fundamentação Teórica}
\label{cap:fundamentacao}

\section{Conceitos de Medição de Energia Elétrica}

A medição de energia elétrica em ambientes industriais é um pilar fundamental para a gestão energética e para o cumprimento de normas como a ISO 50001 \cite{iso50001}. O monitoramento contínuo e preciso de grandezas elétricas permite a identificação de ineficiências, a alocação correta de custos e a otimização do consumo \cite{mamede2017}.

\subsection{Grandezas Elétricas Monitoradas (Potência Ativa, Reativa, Fator de Potência)}

Em sistemas de corrente alternada (CA), a potência elétrica é classificada em três componentes principais, cuja compreensão é essencial para a análise do desempenho energético de uma planta industrial:

\begin{itemize}
    \item \textbf{Potência Ativa ($P$):} Medida em Watts (W) ou quilowatts (kW), é a energia efetivamente convertida em trabalho útil, como o movimento de motores ou o aquecimento \cite{mamede2017}. É a potência que realiza o trabalho e, portanto, a que é faturada pela concessionária de energia.
    \item \textbf{Potência Reativa ($Q$):} Medida em Volt-Ampère Reativo (VAr) ou quilovolt-ampère reativo (kVAr), é a potência necessária para criar e manter os campos eletromagnéticos em equipamentos indutivos, como motores, transformadores e reatores \cite{kosow1982}. Embora não realize trabalho útil, é indispensável para o funcionamento desses equipamentos e sua circulação na rede pode sobrecarregar o sistema.
    \item \textbf{Potência Aparente ($S$):} Medida em Volt-Ampère (VA) ou quilovolt-ampère (kVA), é a soma vetorial da Potência Ativa e da Potência Reativa. Representa a potência total que o sistema de distribuição deve fornecer para atender à demanda de carga \cite{mamede2017}.
\end{itemize}

O \textbf{Fator de Potência ($FP$)} é a razão entre a Potência Ativa e a Potência Aparente ($FP = P/S$). Ele é um indicador crucial da eficiência com que a energia elétrica está sendo utilizada \cite{medeiros2021}. Um baixo fator de potência indica um consumo excessivo de Potência Reativa, o que pode levar a multas por parte da concessionária e a perdas de energia na rede interna da indústria \cite{prado2021}. O monitoramento contínuo dessas grandezas é a base para a implementação de ações de correção e otimização energética.

\subsection{Energia Gerada e Regenerativa em Linhas de Teste}

O conceito de \textbf{energia regenerativa} é amplamente conhecido no setor automotivo, especialmente em veículos elétricos e híbridos, onde o motor elétrico atua como um gerador durante a frenagem ou desaceleração, convertendo a energia cinética do veículo de volta em energia elétrica \cite{azevedo2020}.

No contexto industrial, e especificamente nas linhas de montagem de veículos, a presença de \textbf{linhas de teste} para motores ou veículos completos pode gerar um fenômeno similar. Durante testes de desempenho ou simulações de frenagem, o motor do veículo, quando acoplado a um sistema de dinamômetros ou freios regenerativos, pode funcionar como um gerador, injetando energia de volta na rede elétrica da fábrica \cite{andrade2019}.

A medição e quantificação dessa energia gerada e regenerativa é um diferencial importante para o balanço energético do setor. A capacidade de monitorar o fluxo bidirecional de energia (consumo e geração) permite o cálculo mais preciso do consumo líquido de cada linha de montagem e a identificação de oportunidades para otimizar o aproveitamento dessa energia, alinhando-se aos princípios de sustentabilidade e eficiência energética da ISO 50001 \cite{iso50001}.


\section{Conceitos de Medição de Energia Elétrica}

A medição de energia elétrica em ambientes industriais é um pilar fundamental para a gestão energética e para o cumprimento de normas como a ISO 50001 (ABNT, 2018). O monitoramento contínuo e preciso de grandezas elétricas permite a identificação de ineficiências, a alocação correta de custos e a otimização do consumo (MAMEDE FILHO, 2017).

\subsection{Grandezas Elétricas Monitoradas (Potência Ativa, Reativa, Fator de Potência)}

Em sistemas de corrente alternada (CA), a potência elétrica é classificada em três componentes principais, cuja compreensão é essencial para a análise do desempenho energético de uma planta industrial:

\begin{itemize}
    \item \textbf{Potência Ativa ($P$):} Medida em Watts (W) ou quilowatts (kW), é a energia efetivamente convertida em trabalho útil, como o movimento de motores ou o aquecimento (MAMEDE FILHO, 2017). É a potência que realiza o trabalho e, portanto, a que é faturada pela concessionária de energia.
    
    \item \textbf{Potência Reativa ($Q$):} Medida em Volt-Ampère Reativo (VAr) ou quilovolt-ampère reativo (kVAr), é a potência necessária para criar e manter os campos eletromagnéticos em equipamentos indutivos, como motores, transformadores e reatores (KOSOW, 1982). Embora não realize trabalho útil, é indispensável para o funcionamento desses equipamentos e sua circulação na rede pode sobrecarregar o sistema.
    
    \item \textbf{Potência Aparente ($S$):} Medida em Volt-Ampère (VA) ou quilovolt-ampère (kVA), é a soma vetorial da Potência Ativa e da Potência Reativa. Representa a potência total que o sistema de distribuição deve fornecer para atender à demanda de carga (MAMEDE FILHO, 2017).
\end{itemize}

O \textbf{Fator de Potência ($FP$)} é a razão entre a Potência Ativa e a Potência Aparente ($FP = P/S$). Ele é um indicador crucial da eficiência com que a energia elétrica está sendo utilizada (MEDEIROS, 2021). Um baixo fator de potência indica um consumo excessivo de Potência Reativa, o que pode levar a multas por parte da concessionária e a perdas de energia na rede interna da indústria (PRADO, 2021). O monitoramento contínuo dessas grandezas é a base para a implementação de ações de correção e otimização energética.

\subsection{Energia Gerada e Regenerativa em Linhas de Teste}

O conceito de \textbf{energia regenerativa} é amplamente conhecido no setor automotivo, especialmente em veículos elétricos e híbridos, onde o motor elétrico atua como um gerador durante a frenagem ou desaceleração, convertendo a energia cinética do veículo de volta em energia elétrica (AZEVEDO; GUARNIERI; BOCCATO, 2020).

No contexto industrial, e especificamente nas linhas de montagem de veículos, a presença de \textbf{linhas de teste} para motores ou veículos completos pode gerar um fenômeno similar. Durante testes de desempenho ou simulações de frenagem, o motor do veículo, quando acoplado a um sistema de dinamômetros ou freios regenerativos, pode funcionar como um gerador, injetando energia de volta na rede elétrica da fábrica (ANDRADE, 2019).

A medição e quantificação dessa energia gerada e regenerativa é um diferencial importante para o balanço energético do setor. A capacidade de monitorar o fluxo bidirecional de energia (consumo e geração) permite o cálculo mais preciso do consumo líquido de cada linha de montagem e a identificação de oportunidades para otimizar o aproveitamento dessa energia, alinhando-se aos princípios de sustentabilidade e eficiência energética da ISO 50001 (ABNT, 2018).

\section{Instrumentação Industrial}
\subsection{Multimedidores Siemens Sentron PAC 3200}
\subsection{Controladores Lógicos Programáveis (CLP) Siemens S7-1200}

\section{Comunicação e Sistemas Supervisórios}
\subsection{Protocolo OPC UA}
\subsection{Arquitetura SCADA e Sistemas de Aquisição de Dados (DAQs)}

\section{Indicadores de Desempenho Energético (EnPIs)}

% Conteúdo original do arquivo fundamentacao.tex (mantido como referência para o usuário)
% Após introduzir o contexto da eficiência energética industrial e a necessidade de monitoramento conforme a ISO 50001, esta seção apresenta os conceitos fundamentais relacionados à gestão de energia, aquisição de dados industriais e diagnóstico de falhas em motores elétricos, os quais embasam o desenvolvimento do sistema proposto.

% \section{Gestão e Eficiência Energética}
% A gestão energética tem como objetivo otimizar o uso dos recursos elétricos de modo a reduzir perdas e custos operacionais. No contexto industrial, o consumo elétrico está fortemente associado ao uso de motores e equipamentos de grande porte, o que torna essencial o acompanhamento contínuo de indicadores de desempenho energético.

% A norma ISO 50001 estabelece diretrizes para implementação de sistemas de gestão de energia (SGE), baseados no ciclo PDCA (Planejar, Executar, Verificar e Agir). O cumprimento dessa norma exige medições precisas de consumo, definição de indicadores e documentação de melhorias contínuas no desempenho energético.

% \section{Monitoramento e Aquisição de Dados Industriais}
% O monitoramento contínuo é um dos pilares para atender aos requisitos da ISO 50001. Em ambientes industriais, ele é realizado por dispositivos de medição, como analisadores de energia e controladores programáveis, integrados via protocolos de comunicação.

% Sistemas SCADA (Supervisory Control and Data Acquisition) permitem a supervisão de processos e o registro de variáveis elétricas em tempo real. A integração desses sistemas com tecnologias IoT possibilita o armazenamento e análise remota dos dados energéticos.

% Os principais protocolos utilizados em redes industriais são Modbus, Profinet e OPC UA, que garantem interoperabilidade entre equipamentos de diferentes fabricantes e compatibilidade com plataformas de análise.

% \section{Motores Elétricos e Diagnóstico de Falhas}
% Os motores de indução trifásicos são responsáveis por uma parcela significativa do consumo energético industrial. O desempenho desses motores é diretamente afetado por condições elétricas e mecânicas, sendo o diagnóstico precoce de falhas essencial para evitar paradas não programadas.

% As falhas mais comuns incluem curto-circuitos entre espiras, desgaste de rolamentos, desalinhamentos e desequilíbrios de fase. Técnicas de diagnóstico como a Análise de Assinatura de Corrente (MCSA) e o uso da Transformada Rápida de Fourier (FFT) permitem identificar variações nos sinais de corrente e vibração associadas a essas falhas.

% \section{Instrumentação e Dispositivos de Medição}
% Para o monitoramento elétrico, utilizam-se sensores de corrente, tensão e temperatura, além de medidores inteligentes que consolidam as informações de consumo e qualidade de energia. Dispositivos como o Siemens Sentron PAC3200 possibilitam a leitura detalhada de parâmetros como potência ativa e reativa, fator de potência e distorção harmônica total (THD).

% Esses dados, quando tratados e analisados, permitem tanto o acompanhamento do desempenho energético quanto a identificação de condições anormais em equipamentos, alinhando o sistema às práticas de manutenção preditiva e gestão energética exigidas pela ISO 50001.
