\chapter{Fundamentação Teórica}
\label{cap:fundamentacao}

% PRINCIPAIS PROBLEMAS:
% \begin{itemize}
%     \item Apresentação do embasamento teórico da sua pesquisa não fica clara.
%     \item Cita, mas não indica o que poderia usar de um autor, especialmente metodologias e resultados.
%     \item Não indica resultados e se seriam adequados para seu trabalho
%     \item Muita citação indireta (use mais citações diretas)
%     \item O texto não mostra argumentos e nem posições, só listando outras pesquisas
%     \item CUIDADO!!!!!! Não é só mostrar que leu, fazendo citações que nada têm a ver com a outra – cada citação deve ser coesa com a anterior
%     \item Ter um grau de profundidade dependente da modalidade da pesquisa
%     \item CUIDADO! Apresentação do referencial teórico do projeto é diferente da do trabalho final que poderá mudar com o delineamento do trabalho:
%     \begin{itemize}
%         \item teorias mais aprofundadas
%         \item outros conceitos básicos inter-relacionados
%         \item aspectos mais voltados para o efetivamente mostrado no resultado do seu trabalho com estabelecimento de um “diálogo científico”   
%     \end{itemize}
% \end{itemize}

% SUGESTÃO DE ROTEIRO:
% \begin{enumerate}
%     \item Fundamente os principais conceitos e eventualmente enuncie um MARCO TEÓRICO (um enunciado que é uma dedução ou conceito que será a espinha dorsal do seu trabalho) levado em consideração em todas as seções do projeto; é um conceito “guarda-chuva” do qual não se pretende sair nem se contrapor em nenhum momento. 
%     \item Descrever o que já foi realizado na área específica do estudo (aqui poderia ser uma seção de TRABALHOS RELACIONADOS no artigo) – mostrar como autores antigos e recentes têm tratado o problema ou problemas similares com mais detalhes do que quando citou na apresentação.
%     \item Relacionando os autores, pode-se, especialmente em relação à METODOLOGIA e aos RESULTADOS:
%     \begin{itemize}
%         \item Mostrar o que um complementa em relação ao outro
%         \item O que um é diferente em relação ao outro e qual seria a razão
%         \item As semelhanças de um em relação ao outro e em que isso ajudaria seu trabalho, obtendo ideias das duas pesquisas
%         \item Mas CUIDADO para não relacionar autores TOTALMENTE INCOMPATÍVEIS!!!!!!
%     \end{itemize}
%     \item Situar-se quanto a posição teórica adotada, justificando-a.
%     \item CUIDADO!!!!!! Não é só mostrar que leu, fazendo citações que nada têm a ver com a outra – cada citação deve ser coesa com a anterior
% \end{enumerate}




% \section{Trabalhos Relacionados}
% Teste

% Exemplo de figura. A Figura~\ref{fig:exfig} mostra a Logo do IFPB.

% \begin{figure}[htp]
% 	\centering
% 	\caption{\label{fig:inrush-fig02} Logo IFPB.}
% 	\includegraphics[width = 0.2\linewidth]{images/IFPB.png}
% 	\legend{Fonte: IFPB.}
% 	\label{fig:exfig}
% \end{figure}

% Exemplo de equação: matematicamente, o Fator de Potência (FP) pode ser expresso como:
% \begin{equation}
% 	\label{eq:k-55}
%     {
%     \displaystyle 
%     FP = \frac{\cos(\varphi)}{\sqrt{1 - THD^2}}
%     }
% \end{equation}
% Teste

% \section{Gestão e Eficiência Energética Industrial}
%     PlaceHolder
% \subsection{Conceitos de eficiência e desempenho energético}
%     PlaceHolder
% \subsection{Indicadores energéticos (EnPIs) e análise de consumo}
%     PlaceHolder
% \subsection{Diretrizes e objetivos da ISO 50001}
%     PlaceHolder

% \section{Grandezas Elétricas e Qualidade de Energia}
%     PlaceHolder
% \subsection{Corrente, tensão, potência ativa, reativa e aparente}
%     PlaceHolder
% \subsection{Cálculo e análise em sistemas trifásicos}
%     PlaceHolder
% \subsection{Principais distúrbios elétricos e seus impactos}
%     PlaceHolder

% \section{Sensoriamento e Medição de Grandezas Elétricas}
%     PlaceHolder
% \subsection{Transformadores de corrente e potencial (TC e TP)}
%     PlaceHolder
% \subsection{Medidores digitais e dispositivos inteligentes}
%     PlaceHolder
% \subsection{Aquisição e confiabilidade dos dados de medição}
%     PlaceHolder

% \section{Automação Industrial e Redes de Comunicação}
%     PlaceHolder
% \subsection{Estrutura de sistemas automatizados}
%     PlaceHolder
% \subsection{Protocolos industriais: Modbus, Profinet, Ethernet/IP e OPC UA}
%     PlaceHolder
% \subsection{Integração entre dispositivos de medição e sistemas supervisórios}
%     PlaceHolder

% \section{Arquitetura de Sistemas de Monitoramento Energético}
%     PlaceHolder
% \subsection{Etapas: coleta, processamento, armazenamento e visualização}
%     PlaceHolder
% \subsection{Comunicação e sincronização de dados (WebSocket, MQTT)}
%     PlaceHolder
% \subsection{Sistemas SCADA e IIoT aplicados ao monitoramento energético}
%     PlaceHolder

% \section{Visualização e Análise de Dados Energéticos}
%     PlaceHolder
% \subsection{Dashboards, indicadores e relatórios de consumo}
%     PlaceHolder
% \subsection{Monitoramento em tempo real e análise de tendências}
%     PlaceHolder
% \subsection{Aplicações práticas para gestão e tomada de decisão}
%     PlaceHolder

% \section{Normas e Regulamentações Técnicas Aplicáveis}
%     PlaceHolder
% \subsection{ISO 50001, IEC 61000, NBR 5410 e NR-10}
%     PlaceHolder
% \subsection{Requisitos de segurança, qualidade e eficiência energética}
%     PlaceHolder
% Teste

% \section{Trabalhos e Soluções Correlatas}
% Teste
% \subsection{Sistemas de monitoramento energético existentes}
% Teste
% \subsection{Comparação entre soluções comerciais e acadêmicas}
% Teste
% \subsection{Relevância e diferencial da proposta desenvolvida}
% Teste

Após introduzir o contexto da eficiência energética industrial e a necessidade de monitoramento conforme a ISO 50001, esta seção apresenta os conceitos fundamentais relacionados à gestão de energia, aquisição de dados industriais e diagnóstico de falhas em motores elétricos, os quais embasam o desenvolvimento do sistema proposto.

\section{Gestão e Eficiência Energética}
A gestão energética tem como objetivo otimizar o uso dos recursos elétricos de modo a reduzir perdas e custos operacionais. No contexto industrial, o consumo elétrico está fortemente associado ao uso de motores e equipamentos de grande porte, o que torna essencial o acompanhamento contínuo de indicadores de desempenho energético.

A norma ISO 50001 estabelece diretrizes para implementação de sistemas de gestão de energia (SGE), baseados no ciclo PDCA (Planejar, Executar, Verificar e Agir). O cumprimento dessa norma exige medições precisas de consumo, definição de indicadores e documentação de melhorias contínuas no desempenho energético.

\section{Monitoramento e Aquisição de Dados Industriais}
O monitoramento contínuo é um dos pilares para atender aos requisitos da ISO 50001. Em ambientes industriais, ele é realizado por dispositivos de medição, como analisadores de energia e controladores programáveis, integrados via protocolos de comunicação.

Sistemas SCADA (Supervisory Control and Data Acquisition) permitem a supervisão de processos e o registro de variáveis elétricas em tempo real. A integração desses sistemas com tecnologias IoT possibilita o armazenamento e análise remota dos dados energéticos.

Os principais protocolos utilizados em redes industriais são Modbus, Profinet e OPC UA, que garantem interoperabilidade entre equipamentos de diferentes fabricantes e compatibilidade com plataformas de análise.

\section{Motores Elétricos e Diagnóstico de Falhas}
Os motores de indução trifásicos são responsáveis por uma parcela significativa do consumo energético industrial. O desempenho desses motores é diretamente afetado por condições elétricas e mecânicas, sendo o diagnóstico precoce de falhas essencial para evitar paradas não programadas.

As falhas mais comuns incluem curto-circuitos entre espiras, desgaste de rolamentos, desalinhamentos e desequilíbrios de fase. Técnicas de diagnóstico como a Análise de Assinatura de Corrente (MCSA) e o uso da Transformada Rápida de Fourier (FFT) permitem identificar variações nos sinais de corrente e vibração associadas a essas falhas.

\section{Instrumentação e Dispositivos de Medição}
Para o monitoramento elétrico, utilizam-se sensores de corrente, tensão e temperatura, além de medidores inteligentes que consolidam as informações de consumo e qualidade de energia. Dispositivos como o Siemens Sentron PAC3200 possibilitam a leitura detalhada de parâmetros como potência ativa e reativa, fator de potência e distorção harmônica total (THD).

Esses dados, quando tratados e analisados, permitem tanto o acompanhamento do desempenho energético quanto a identificação de condições anormais em equipamentos, alinhando o sistema às práticas de manutenção preditiva e gestão energética exigidas pela ISO 50001.