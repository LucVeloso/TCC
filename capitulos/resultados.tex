\chapter{Análise e Resultados do Monitoramento Energético}
\label{cap:resultados}

\section{Coleta e Tratamento dos Dados}

O processo de análise e apresentação dos resultados do monitoramento energético foi conduzido com rigor técnico e ético, em conformidade com as políticas de confidencialidade da Stellantis Goiana. Dada a natureza estratégica dos dados de consumo industrial, que podem revelar informações sensíveis sobre a capacidade produtiva e a eficiência operacional da planta, optou-se pela \textbf{normalização e anonimização} de valores absolutos para fins de publicação acadêmica. Esta abordagem garante a preservação da confidencialidade comercial da empresa, ao mesmo tempo em que permite a demonstração clara das tendências, padrões e eficácia do sistema de monitoramento desenvolvido. Os resultados apresentados, portanto, refletem as proporções e variações reais do desempenho energético, sem expor dados brutos que possam comprometer a segurança da informação da organização.

\subsection{Período de Monitoramento e Volume de Dados}

O monitoramento energético foi realizado durante o período de [INSERIR PERÍODO - Ex: janeiro a março de 2025], totalizando [INSERIR DURAÇÃO] de coleta contínua de dados. Durante este intervalo, o sistema registrou um volume significativo de informações, com [INSERIR NÚMERO] de variáveis elétricas provenientes dos 17 pares de multimedidores PAC 3200 instalados nas linhas de montagem. A frequência de coleta de 1 segundo, conforme detalhado na Metodologia, gerou uma base de dados robusta para análises de curto e longo prazo.

\subsection{Metodologia de Análise}

A análise dos dados coletados foi pautada na identificação de padrões de consumo, na avaliação da eficiência energética e na detecção de anomalias. As principais metodologias empregadas incluíram:

\begin{itemize}
    \item \textbf{Análise de Tendências:} Utilização de gráficos de séries temporais para visualizar o comportamento do consumo de energia (kWh) ao longo do dia, semana e mês, com destaque para a separação por turnos de trabalho.
    \item \textbf{Comparação de Períodos:} Análise comparativa do consumo entre diferentes turnos, dias da semana e períodos de produção versus não-produção (manutenção, feriados), visando identificar oportunidades de otimização.
    \item \textbf{Cálculo de EnPIs Normalizados:} Implementação do indicador kWh/carro, ajustado para refletir a eficiência energética por unidade produzida, permitindo uma avaliação justa do desempenho independentemente das variações no volume de produção.
    \item \textbf{Análise de Qualidade de Energia:} Monitoramento do Fator de Potência e da Distorção Harmônica Total (THD) para identificar problemas que possam afetar a eficiência e a vida útil dos equipamentos.
\end{itemize}

\section{Resultados do Monitoramento por Linha de Montagem}
\subsection{Análise do Consumo de Energia (kWh)}
\subsection{Desempenho do Fator de Potência}

\section{Análise da Energia Regenerativa}
\subsection{Quantificação da Energia Gerada nas Linhas de Teste}
\subsection{Impacto no Balanço Energético do Setor}

\section{Avaliação do Sistema em Relação à ISO 50001}
\subsection{Atendimento aos Requisitos de Medição e Rastreabilidade}
\subsection{Utilidade para a Manutenção e Tomada de Decisão}
