\chapter{Introdução}\label{Introdução}

A crescente necessidade de aprimorar a eficiência energética industrial tem levado as organizações a adotar metodologias e sistemas de gestão baseados em normas internacionais, como a ISO 50001, que estabelece diretrizes para o desenvolvimento, implementação e melhoria contínua de sistemas de gestão de energia. Essa norma orienta as empresas a monitorar, medir e analisar o consumo energético de forma sistemática, permitindo identificar oportunidades de otimização e redução de custos operacionais sem comprometer a produtividade. Segundo a ABNT NBR ISO 50001:2018, a aplicação de um sistema de gestão de energia visa “habilitar uma organização a seguir uma abordagem sistemática para alcançar a melhoria contínua do desempenho energético, incluindo eficiência, uso e consumo de energia” (ABNT, 2018).

No cenário industrial brasileiro, a busca por maior eficiência energética também se apresenta como uma necessidade estratégica diante do crescimento das demandas de produção e do aumento dos custos com energia elétrica. De acordo com o Programa Nacional de Conservação de Energia Elétrica (PROCEL), o setor industrial representa uma das maiores parcelas do consumo energético nacional, e a adoção de práticas sistemáticas de gestão e monitoramento pode reduzir em até 20% o desperdício de energia em determinados processos produtivos (EPE, 2022). Essa realidade reforça a importância de soluções tecnológicas que aliem baixo custo de implementação com alto potencial de diagnóstico e melhoria contínua.

Definição do problema e motivação

O principal desafio enfrentado por muitas indústrias é atender às exigências de rastreabilidade e controle energético impostas pela ISO 50001, sem incorrer em altos custos de implantação de novos sistemas. Em muitos casos, há equipamentos de medição e comunicação já instalados na planta, porém subutilizados devido à falta de integração entre dispositivos, protocolos e plataformas de supervisão.

Surge, portanto, a necessidade de propor uma solução que aproveite a infraestrutura existente, integrando dados de medidores e sensores industriais em um sistema único de monitoramento energético. Além de atender às normas de gestão, essa integração pode ampliar a capacidade de diagnóstico técnico, auxiliando equipes de manutenção na identificação de falhas, anomalias e comportamentos anormais de consumo.

\section{Objetivos da pesquisa}

O objetivo geral deste trabalho é desenvolver um sistema de monitoramento energético industrial baseado na integração de equipamentos e dados já disponíveis na planta, com o propósito de atender aos requisitos da ISO 50001 e fornecer suporte técnico para manutenção e análise de desempenho.

Os objetivos específicos são:

Mapear os pontos de medição existentes na instalação e suas respectivas variáveis elétricas e energéticas;

Integrar os dispositivos inteligentes (medidores, sensores e controladores) por meio de protocolos industriais de comunicação;

Desenvolver um sistema supervisório para coleta, armazenamento e visualização de dados em tempo real;

Implementar indicadores energéticos (EnPIs) conforme as diretrizes da ISO 50001;

Avaliar o desempenho do sistema quanto à confiabilidade, precisão e utilidade para análise de falhas e manutenção preventiva.

% Please add the following required packages to your document preamble:
% \usepackage[table,xcdraw]{xcolor}
% If you use beamer only pass "xcolor=table" option, i.e. \documentclass[xcolor=table]{beamer}

%%%%%%%%%%%%%%%%%%%%%%%%%%%%%%%%%%%%%%%%%%%%%%%%%
% Capitulo de exemplos utilizando arquivo externo 
%%%%%%%%%%%%%%%%%%%%%%%%%%%%%%%%%%%%%%%%%%%%%%%%%
%\include{abntex-exemplos} % comente esta linha para facilitar, mas não apague o arquivo abntex-exemplos.tex pois ele contém exemplos interessantes que podem auxiliar na elaboração da dissertação.
